\documentclass[12pt, twoside]{article}
\usepackage{jmlda}
\newcommand{\hdir}{.}

\begin{document}

\title
    %[Шаблон статьи для публикации] % краткое название; не нужно, если полное название влезает в~колонтитул
    {Моделирование динамики физических систем с помощью Physics-Informed Neural Networks}
\author
    %[И.\,О.~Автор] % список авторов (не более трех) для колонтитула; не нужен, если основной список влезает в колонтитул
    {А.\,А.~Терентьев$^1$}  %основной список авторов, выводимый в оглавление
   %[И.\,О.~Автор$^1$, И.\,О.~Соавтор$^2$
   %[И.\,О.~Фамилия$^{1,2}$] % список авторов, выводимый в заголовок; не нужен, если он не отличается от основного
\email
    {terentev.aa@phystech.edu}
%\thanks
    %{
     %Работа выполнена при
     %частичной
     %финансовой поддержке РФФИ, проекты \No\ \No 00-00-00000 и %00-00-00001.
     %}
\organization
    {$^1$ФПМИ, МФТИ}
\abstract
    { В работе решается задача выбора оптимальной модели в классе нейронных сетей для предсказания динамики физической системы. Использование классических нейросетевых методов для построения траекторий физических систем не позволяет получить решение, удовлетворяющее законам сохранения. Для их учёта используются различные модификации прямого решения с помощью нейронных сетей, которые бы описывали Лагранжеву динамику и опирались на теорему Нётер. В данной работе предлагается исследовать модели на основе Нётеровских сетей с более сложной архитектурой внутренних слоёв, а также рассмотреть адекватность моделирования систем, подвергающихся внешнему воздействию.
	
\bigskip
\noindent
%\textbf{Ключевые слова}: \emph {ключевое слово; ключевое слово; еще ключевые слова, разделенные <<;>>}
}


%данные поля заполняются редакцией журнала
%\doi{10.21469/22233792}
%\receivedRus{28.02.2023}
%\receivedEng{January 01, 2017}

\maketitle
%\linenumbers

\section{Введение}
Одной из основных задач механики является задача моделирования динамики различных систем. Эта задача находит применение в различных областях: моделирование траектории, работа механизмов или систем, их реакции на внешние и внутренние воздействия, которые могут привести к нарушению работы или полному выводу их из строя.

Для того, чтобы использовать лагранжеву механику необходимо выбрать обобщенные координаты, которые полностью бы описывали поведения системы, иметь знания о Лагранжиане системы, который определяется как разница между кинетической энергией (T)\ и потенциальной энергией (V) 

\[L\ =\ T\ -\ V\]

\noindent
и величине обобщенных сил системы в случае внешних воздействий.
Динамика системы в таком случае описывается системой уравнений Эйлера-Лагранжа:

\[ \frac{\partial L}{\partial x}\ -\ \frac{d}{dt}\ \frac{\partial L}{\partial\dot{x}}=Q_x=0.\]

\noindent
Мы не всегда можем иметь представление о лагранжиане в случае сложных систем. 

 Получить решение данной задачи с помощью классической Ньютоновской механики и аппроксимации дифференциальных уравнений нейронными сетями практически не представляется возможным для хаотических систем, таких как двойной маятник, т.к. такие сети не имеют знания о законах сохранения, импульса, момента импульса и других законах физики, и оказываются неспособными получить оптимальные параметры модели. 
 
Одним из способов его нахождения являются Лагранжевы нейронные сети (LNN), которые аппроксимируют лагранжиан системы, после чего оказывается возможным возможным решить возникающие уравнения Эйлера-Лагранжа. Таким образом, нейронная сеть оказывается способной учесть физику системы, а точнее о закон сохранения энергии. Для некоторых систем, используя теорему Нётер и некоторых знаниях о симметриях Лагранжиана системы, могут быть получены более точные решения, учитывающие больше законов сохранения. Например, Нётеровские Лагранжевы нейронные сети (NLNN) способны учитывать трансляционную и вращательную симметрию лагранжиана и, следовательно, находят оптимальное решение для систем, в которых выполняются законы сохранения импульса и момента импульса.

На точность решения влияет не только количество учтенных интегралов движения, но и выбор самой модели, на основе которой находится соответствующая функция динамики системы. В работе исследуются различные архитектуры нейронных сетей в основе NLNN.

Для инженерных задач необходимо не только уметь предсказывать динамику самой системы, но и её реакцию на внешнее воздействие. Так, для гироскопических приборов критическим является их реакция на колебательные воздействия, которые могут приводить к погрешностям или полному выходу системы из строя, причем необходимо исследовать поведение прибора при разных эксплуатационных условиях.

Для вычислительного эксперимента соответственно взята модель гироскопа. В данной системе сохраняется энергия, импульс и момент импульса. Поведение системы исследовалось с помощью NLNN, в том числе при различных внешних воздействиях. Результаты моделирования сравнивались с траекториями, полученными аналитически с помощью метода Рунге-Кутты 4-ого порядка.

\section{Постановка задачи регрессии динамики физической системы}
%%%%%%%%%%%%%%%%%%%%%%%%%%%%%%%%%%%%%%%%%%%%%%%%%%%%%%%%%%%%%%%%%%%%%%%%%%
Задачу моделирования динамики системы можно свести к задаче регрессии.
Пусть дана выборка из $m$ траекторий 
$$\left\{\mathbf{x}_i, \mathbf{y}_i\right\}_{i=1}^m,$$ где $\mathbf{x}_i = (\mathbf{q}_i, \mathbf{\dot{q}}_i)$~--  координаты траектории движения двойного маятника, $\mathbf{{y}}_i = \mathbf{\dot{x}}_i = (\mathbf{\dot{q}}_i, \mathbf{\ddot{q}}_i)$ -- динамика движения системы двойного маятника, $\mathbf{q}_i \in \mathbb{R}^{r \times n}$, где $r$ -- количество координат, $n$ -- длина траектории.

Регрессионная модель выбирается из класса нейронных сетей
$$\{\mathbf{f}_k\colon(\mathbf{w}, \mathbf{X})\to  \hat{\mathbf{y}} \mid k \in \mathcal{K}\},$$ где $\mathbf{w} \in \mathbb{W}$~-- параметры модели, $\hat{\mathbf{y}} = \mathbf{f} (\mathbf{X},\mathbf{w}) \in \mathbb{R}^{2\times r \times n}, \mathbf{X} = \bigcup_{i=1}^m \mathbf{x}_i$.

В качестве функция ошибки взята квадратичная ошибка: 
$$\mathcal{L}(\mathbf{y}, \mathbf{X}, \mathbf{w}) =\left\lVert \hat{\mathbf{y}} - \mathbf{y} \right\rVert^{2}_2.$$

Таким образом, задача моделирования динамики системы представлена в виде задачи минимизации квадратичной ошибки: 
$$\textbf{w}^* = \underset{\mathbf{w}\in\mathbb{W}}{\text{argmin}}\bigl(\mathcal{L}(\textbf{w})\bigr).$$
\paragraph{Название параграфа}
Разделы и~параграфы, за исключением списков литературы, нумеруются.

\section{Заключение}
Желательно, чтобы этот раздел был, причём он не~должен дословно повторять аннотацию.
Обычно здесь отмечают, каких результатов удалось добиться, какие проблемы остались открытыми.

%%%% если имеется doi цитируемого источника, необходимо его указать, см. пример в \bibitem{article}
%%%% DOI публикации, зарегистрированной в системе Crossref, можно получить по адресу http://www.crossref.org/guestquery/
\begin{thebibliography}{99}
\bibitem{book}
    \BibAuthor{Гуссенс~М., Миттельбах~Ф., Cамарин~А.}
    \BibTitle{Путеводитель по пакету \LaTeX\ и~его расширению \LaTeXe} / Пер. с англ.~---
    М.:~Мир, 1999. 606~с.
    (\BibAuthor{Goossens M., Mittelbach F., Samarin A.}
     \BibTitle{The \LaTeX\ companion}.~--- 2nd ed.~--- Reading, MA, USA: Addison-Wesley, 1994. 528 p.)

\bibitem{article}
    \BibAuthor{Загуренко~А.\,Г., Коротовских~В.\,А., Колесников~А.\,А., Тимонов~А.\,В., Кардымов~Д.\,В.}
    Технико-экономическая оптимизация дизайна гидроразрыва пласта~//
    \BibJournal{Нефтяное хозяйство}, 2008. Т.~11. \No\,1. С.~54--57.
	\BibDoi{10.3114/S187007708007}.

\bibitem{webArticle}
	\BibAuthor{Blaga~P.\,A.}
	Commutative Diagrams with XY-pic II. Frames and Matrices~//
	\BibJournal{PracTEX J.}, 2007. Vol.\,4.
	URL: \BibUrl{https://tug.org/pracjourn/2007-1/blaga/blaga.pdf}.

\bibitem{webResource}
	XYpic.
	URL: \BibUrl{http://akagi.ms.u-tokyo.ac.jp/input9.pdf}.
	
\bibitem{inproceedingsRus}
	\BibAuthor{Усманов~Т.\,С., Гусманов~А.\,А., Муллагалин~И.\,З., Мухаметшина~Р.\,Ю., Червякова~А.\,Н., Свешников~А.\,В.}
	Особенности проектирования разработки месторождений с применением гидроразрыва пласта~//
	\BibJournal{Труды 6-го Междунар. симп. <<Новые ресурсосберегающие технологии недропользования и повышения нефтегазоотдачи>>}.~---
	М.:~Издательство, 2007. С.~267--272.

\bibitem{inproceedingsEng}
    \BibAuthor{Author~N.}
    Paper title~//
    \BibJournal{10th Conference (International) on Any Science Proceedings}.~---
    Place of publication: Publisher, 2009. P.~111--122.

\bibitem{techreport}
	\BibAuthor{Lambert~P.}
  	\BibTitle{The title of the work}.
  	Place of publication:~The institution that published, 1993.  Report~2.
 	
\end{thebibliography}

%%%% если имеется doi цитируемого источника, необходимо его указать, см. пример в \bibitem{article}
%%%% DOI публикации, зарегистрированной в системе Crossref, можно получить по адресу http://www.crossref.org/guestquery/.


\end{document}
